\documentclass{ltjsarticle}
\usepackage{luatexja}

\usepackage{amsmath, amssymb, latexsym, physics, mathrsfs, bm}
\usepackage{graphicx}
\usepackage{grffile}
\usepackage[svgnames]{xcolor}
\usepackage[
     colorlinks        = true,
     unicode           = true,
     pdfstartview      = FitV,
     linktocpage       = true,
     linkcolor         = OrangeRed,
     citecolor         = MediumSeaGreen,
     urlcolor          = RoyalBlue,
     bookmarks         = true,
     bookmarksnumbered = true,
     pdftitle={いい感じのタイトル},
     pdfauthor={筒井翔一朗}
]{hyperref}


\begin{document}
\title{いい感じのタイトル}
\author{筒井翔一朗}

\maketitle
\tableofcontents

\section{これは節}
これは参考文献~\cite{Hubbard.PhysRevLett.3.77}です。
これは\href{https://dlmf.nist.gov/}{DLMF}のリンクです
\footnote{
     これは脚注です。
}。

\section{これも節}
けものはいてものけものはいない。
\begin{align}
     \int_0^\pi \dd{x} \sin x  \label{sin}
\end{align}
これは積分です~\eqref{sin}。
\begin{figure}[b]
     \centering
     \includegraphics[width=0.5\linewidth]{tiger}
     \caption{トラのやつ}
\end{figure}

\begin{table}[htb]
	\centering
	\begin{tabular}{ccc} \hline
          col1 & col2 & col3 \\ \hline
          $a_{11}$ & $a_{12}$ & $a_{13}$ \\
          $a_{21}$ & $a_{22}$ & $a_{23}$ \\
          $a_{31}$ & $a_{32}$ & $a_{33}$ \\ \hline
	\end{tabular}
	\caption{いい感じの表}
\end{table}



\bibliography{sample.bib}
\bibliographystyle{h-physrev5}

\end{document}